\PassOptionsToPackage{eulerchapternumbers,listings, pdfspacing, subfig,beramono,thesispaper}{classicthesis}					


%\usepackage{mathptmx}                  % use matching math font
%\usepackage{times}                     % we use Times as the main font
%\renewcommand*\ttdefault{txtt}         % a nicer typewriter font
%\usepackage{tabu}                      % only used for the table example
%\usepackage{amsfonts, amssymb} % used for mathbb{}
%\usepackage{booktabs}                  % only used for the table example
%\usepackage[utf8]{inputenc}
%\usepackage[english]{babel}

\usepackage{ifthen}
\newboolean{enable-backrefs}
\setboolean{enable-backrefs}{false}

% Personal data
\newcommand{\myTitle}{Explaining Projections of High-Dimensional Data \xspace}
\newcommand{\mySubtitle}{~\xspace}
\newcommand{\myName}{Zonglin Tian\xspace}
\newcommand{\myUni}{Utrecht University\xspace}

% Setup and useful commands
\newcounter{dummy} 
\newlength{\abcd} 
\providecommand{\mLyX}{L\kern-.1667em\lower.25em\hbox{Y}\kern-.125emX\@}
\newcommand{\abbr}[1]{\textsc{\MakeLowercase{#1}}}

\newlength{\figureHalf}\setlength{\figureHalf}{139.74pt}
\newlength{\figureThird}\setlength{\figureThird}{90.32pt}
\newlength{\figureFourth}\setlength{\figureFourth}{65.61pt}

\newlength{\figureHalfBigSkip}\setlength{\figureHalfBigSkip}{129.8pt}
\newlength{\figureThirdBigSkip}\setlength{\figureThirdBigSkip}{77.0667pt}
\newlength{\figureFourthBigSkip}\setlength{\figureFourthBigSkip}{50.7pt}

% More packages
\usepackage{verbatim}
\usepackage{printlen}
\usepackage{makeidx}

\usepackage[lining]{libertine}
\usepackage[T1]{fontenc}
\usepackage{textcomp}
\usepackage{amsmath}
\usepackage{amsthm}
%\usepackage[libertine]{newtxmath}
\usepackage{mathtools}
%\usepackage[bb=pazo,bbscaled=0.94,cal=cm,calscaled=0.96]{mathalfa}
\usepackage{bm}
\usepackage{rotating}
\usepackage{mdframed}			% Create frames


%IVAPP23
\usepackage{epsfig}
%\usepackage{subcaption}
\usepackage{calc}
%\usepackage{amssymb}
\usepackage{amstext}
\usepackage{amsmath}
\usepackage{amsthm}
\usepackage{multicol}
%\usepackage{pslatex}
\usepackage{url}
%\usepackage{apalike}
\usepackage[bottom]{footmisc}

% \usepackage[ruled,noline,noend]{algorithm2e}
\usepackage{algorithm}
\usepackage{algorithmic}

\renewcommand{\algorithmicrequire}{\textbf{Input:}}
\renewcommand{\algorithmicensure}{\textbf{Output:}}

\usepackage{sidecap}
\usepackage{marginnote}
\usepackage{tikz}
\usetikzlibrary{arrows,calc,positioning}
\usepackage{tabulary}
\usepackage{enumitem}

\newenvironment{myabstract}%
{\textsc{abstract:} \slshape\footnotesize }%
{}

\PassOptionsToPackage{utf8}{inputenc}	
 \usepackage{inputenc}				

\PassOptionsToPackage{dutch,english,portuguese}{babel}
 \usepackage{babel}					

% \PassOptionsToPackage{square,numbers,sort&compress}{natbib}
\PassOptionsToPackage{round}{natbib}
\usepackage{natbib}				

\usepackage{scrhack} % fix warnings when using KOMA with listings package          
\usepackage{xspace} % to get the spacing after macros right  
\usepackage{mparhack} % get marginpar right
\PassOptionsToPackage{printonlyused,smaller}{acronym}
	\usepackage{acronym} 
\ifdefined\bflabel
	\renewcommand{\bflabel}[1]{{#1}\hfill} % fix the list of acronyms
\else
\fi


% Setup floats: tables, (sub)figures, and captions
\usepackage{tabularx} % better tables
	\setlength{\extrarowheight}{3pt} % increase table row height
\newcommand{\tableheadline}[1]{\multicolumn{1}{c}{\spacedlowsmallcaps{#1}}}
\newcommand{\myfloatalign}{\centering} % to be used with each float for alignment
\usepackage{caption}
\captionsetup{format=hang,font=small}
\usepackage{subfig}  

%Setup code listings
\usepackage{listings} 
\lstset{language=[LaTeX]Tex,%C++,
    keywordstyle=\color{RoyalBlue},%\bfseries,
    basicstyle=\small\ttfamily,
    %identifierstyle=\color{NavyBlue},
    commentstyle=\color{Green}\ttfamily,
    stringstyle=\rmfamily,
    numbers=none,%left,%
    numberstyle=\scriptsize,%\tiny
    stepnumber=5,
    numbersep=8pt,
    showstringspaces=false,
    breaklines=true,
    frameround=ftff,
    frame=single,
    belowcaptionskip=.75\baselineskip
    %frame=L
} 

% PDFLaTeX, hyperreferences and citation backreferences
\PassOptionsToPackage{pdftex,hyperfootnotes=false,pdfpagelabels,breaklinks}{hyperref}
	\usepackage{hyperref}
\pdfcompresslevel=9
\pdfadjustspacing=1 
\PassOptionsToPackage{pdftex}{graphicx}
	\usepackage{graphicx} 
\usepackage[capitalise]{cleveref}

% Setup the style of the backrefs from the bibliography
\newcommand{\backrefnotcitedstring}{\relax}%(Not cited.)
\newcommand{\backrefcitedsinglestring}[1]{(Cited on page~#1.)}
\newcommand{\backrefcitedmultistring}[1]{(Cited on pages~#1.)}
\ifthenelse{\boolean{enable-backrefs}}%
{%
		\PassOptionsToPackage{hyperpageref}{backref}
		\usepackage{backref} % to be loaded after hyperref package 
		   \renewcommand{\backreftwosep}{ and~} % separate 2 pages
		   \renewcommand{\backreflastsep}{, and~} % separate last of longer list
		   \renewcommand*{\backref}[1]{}  % disable standard
		   \renewcommand*{\backrefalt}[4]{% detailed backref
		      \ifcase #1 %
		         \backrefnotcitedstring%
		      \or%
		         \backrefcitedsinglestring{#2}%
		      \else%
		         \backrefcitedmultistring{#2}%
		      \fi}%
}{\relax}    

% Hyperreferences
\hypersetup{%
    colorlinks=true, linktocpage=true, pdfstartpage=3, pdfstartview=FitV,%
    breaklinks=true, pdfpagemode=UseNone, pageanchor=true, pdfpagemode=UseOutlines,%
    plainpages=false, bookmarksnumbered, bookmarksopen=true, bookmarksopenlevel=1,%
    hypertexnames=true, pdfhighlight=/O,
    urlcolor=black, linkcolor=black, citecolor=black, 
    pdftitle={\myTitle},%
    pdfauthor={\textcopyright\ \myName},%
    pdfsubject={},%
    pdfkeywords={},%
    pdfcreator={pdfLaTeX},%
    pdfproducer={LaTeX with hyperref and classicthesis}%
}   

% Setup autoreferences
\makeatletter
\@ifpackageloaded{babel}%
    {%
       \addto\extrasenglish{%
					\renewcommand*{\figureautorefname}{Figure}%
					\renewcommand*{\tableautorefname}{Table}%
					\renewcommand*{\partautorefname}{Part}%
					\renewcommand*{\chapterautorefname}{Chapter}%
					\renewcommand*{\sectionautorefname}{Section}%
					\renewcommand*{\subsectionautorefname}{Section}%
					\renewcommand*{\subsubsectionautorefname}{Section}% 	
				}%
       \addto\extrasngerman{% 
					\renewcommand*{\paragraphautorefname}{Absatz}%
					\renewcommand*{\subparagraphautorefname}{Unterabsatz}%
					\renewcommand*{\footnoteautorefname}{Fu\"snote}%
					\renewcommand*{\FancyVerbLineautorefname}{Zeile}%
					\renewcommand*{\theoremautorefname}{Theorem}%
					\renewcommand*{\appendixautorefname}{Anhang}%
					\renewcommand*{\equationautorefname}{Gleichung}%        
					\renewcommand*{\itemautorefname}{Punkt}%
				}%	
			\providecommand{\subfigureautorefname}{\figureautorefname}%  			
    }{\relax}
\makeatother


\listfiles

\usepackage{classicthesis} 

% Further adjustments (experimental)
\usepackage{bibentry}
\nobibliography*

\providecommand{\doi}[1]{\abbr{DOI} \href{http://dx.doi.org/#1}{\urlstyle{same}\nolinkurl{#1}}}

\emergencystretch=1em

\newtheoremstyle{theoremStyle}
    {\topsep}                    % Space above
    {\topsep}                    % Space below
    {\itshape}                   % Body font
    {}                           % Indent amount
    {\scshape}                   % Theorem head font
    {.}                          % Punctuation after theorem head
    {.5em}                       % Space after theorem head
    {}  % Theorem head spec (can be left empty, meaning ‘normal’)
\theoremstyle{theoremStyle}
\newtheorem{theorem}{Theorem}
\newtheorem{lemma}[theorem]{Lemma}
\newtheorem{corollary}[theorem]{Corollary}
\newtheorem{proposition}[theorem]{Proposition}

\newtheoremstyle{exampleStyle}
    {0}                          % Space above
    {0}                          % Space below
    {}                           % Body font
    {}                           % Indent amount
    {\scshape}                   % Theorem head font
    {.}                          % Punctuation after theorem head
    {.5em}                       % Space after theorem head
    {}  % Theorem head spec (can be left empty, meaning ‘normal’)
\theoremstyle{exampleStyle}
\newtheorem{example}[theorem]{Example}
\usepackage{mdframed}
\surroundwithmdframed[bottomline=false,topline=false,rightline=false,
	linewidth=1pt,
	linecolor=gray,
	%innerleftmargin=7pt,
	skipabove=\topsep,
	skipbelow=0,
	innertopmargin=0,
	innerbottommargin=0]{example}

% Counters
\numberwithin{equation}{chapter}
\numberwithin{theorem}{chapter}
\numberwithin{figure}{chapter}

% CRef names
\crefname{algocf}{Alg.}{Algs.}
\Crefname{algocf}{Algorithm}{Algorithms}

\setlength{\marginparwidth}{8em}%

% Personal commands
\newcommand{\makenote}[1]{{ \color{red} [#1]}}

\DeclareMathOperator*{\argmax}{arg\,max}
\DeclareMathOperator*{\argmin}{arg\,min}

\newcommand{\defaultwidth}{width=}

\DeclareMathOperator{\cov}{cov}
\DeclareMathOperator{\corr}{corr}
\DeclareMathOperator{\std}{std}
\DeclareMathOperator{\var}{var}
\DeclareMathOperator{\Val}{Val}

\DeclareMathOperator*{\KL}{KL}
\newcommand{\bs}{\bm}

\newcommand\independent{\protect\mathpalette{\protect\independenT}{\perp}}
\def\independenT#1#2{\mathrel{\rlap{$#1#2$}\mkern2mu{#1#2}}}

%% -----------------------------------------------
% My own commands
%% -----------------------------------------------

\newcommand{\fold}[1]{\mathcal{#1}}
\newcommand{\set}[1]{\mathbb{#1}}
\newcommand{\vecx}[1]{\boldsymbol{#1}}
\newcommand{\state}{\vecx{\phi}}

\newcommand{\chref}[1]{\autoref{ch:#1}}
\newcommand{\secref}[1]{\autoref{sec:#1}}
\newcommand{\figref}[1]{\autoref{fig:#1}}
\newcommand{\eqnref}[1]{\autoref{eqn:#1}}
\newcommand{\tabref}[1]{\autoref{tab:#1}}
\newcommand{\quadref}[1]{\autoref{quad:#1}}
\newcommand{\parref}[1]{\autoref{par:#1}}	
\newcommand{\coderef}[1]{\autoref{code:#1}}	
\newcommand{\appref}[1]{\autoref{app:#1}}
\newcommand{\itemref}[1]{\autoref{item:#1}}

\hypersetup{
	pdftitle={\myTitle}, 
	pdfauthor={\myName},
	pdfsubject={},
	pdfcreator={},
	pdfkeywords={}, 
	colorlinks=true,       		% false: boxed links; true: colored links
	linkcolor=blue,          	% color of internal links
	citecolor=blue,        		% color of links to bibliography
	filecolor=magenta,      	% color of file links
	urlcolor=blue,
	bookmarksdepth=4,
	plainpages = false,
  	linktocpage
}

% Just right frame
\newmdenv[leftline=false, topline=false, bottomline=false]{rightframe}

% Left and bottom frame
\newmdenv[topline=false, rightline=false]{leftbot}

% Top and bottom frame
\newmdenv[rightline=false, leftline=false]{topbot}

\newcommand{\hyp}[1]{
	\begin{topbot}
		\vspace{.5cm}	
		% \setlength{\parindent}{4em}
		% \setlength{\parskip}{2em}
		\noindent
		\emph{Hyphotesis: ``{#1}''}.
		\vspace{.5cm}
	\end{topbot}
}

\newcommand{\eg}{\emph{e.g.}}
\newcommand{\ie}{\emph{i.e.}}
